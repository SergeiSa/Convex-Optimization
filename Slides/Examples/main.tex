\documentclass{beamer}

\input{settings.tex}


\title{Examples}
\subtitle{\mycoursetitle, lecture ??}
\author{by Sergei Savin}
\centering
\date{\mydate}



\begin{document}
\maketitle


\begin{frame}{Content}

\begin{itemize}
\item  Converting State-Space to ODE
\end{itemize}

\end{frame}




\begin{frame}{Inverse Dynamics, 1}
	%\framesubtitle{How do we know the state?}
	\begin{flushleft}
		
		Consider a dynamical system:
		%
		\begin{equation}
			\bo{H} \ddot{\bo{q}} + \bo{C}\dot{\bo{q}} + \bo{g} = \bo{B}\bo{u}
		\end{equation}		
		
		\textbf{Task 1}: 
		For the current state $\bo{q}$, $\dot{\bo{q}}$ find such $\bo{u}$ (if it exists) that $\ddot{\bo{q}} = \bo{a}$.
		
		\bigskip
		
		\textbf{Solution}. The solution will exist if the vector $\bo{r} = \bo{H} \bo{a} + \bo{C}\dot{\bo{q}} + \bo{g} $ lies in the column space on $\bo{B}$. The condition for the vector to be in the column space of $\bo{B}$ is:
		%
		\begin{equation}
			(\bo{I} - \bo{B}\bo{B}^+)\bo{r} = 0
		\end{equation}		
		%
		If the solution exists, the it takes form:
		%
		\begin{equation}
			\bo{u} = \bo{B}^+ (\bo{H} \bo{a} + \bo{C}\dot{\bo{q}} + \bo{g} )
		\end{equation}		
		
	\end{flushleft}
\end{frame}




\begin{frame}{Inverse Dynamics, 2}
	%\framesubtitle{How do we know the state?}
	\begin{flushleft}
		
		\textbf{Task 2}: 
		For the current state $\bo{q}$, $\dot{\bo{q}}$ find if there exist multiple controls $\bo{u}$ that make $\ddot{\bo{q}} = \bo{a}$. If yes, find all of them.
		
		\bigskip
		
		\textbf{Solution}. If $(\bo{I} - \bo{B}\bo{B}^+)\bo{r} = 0$ then at least one solution to the ID exists. Multiple solutions will exist iff $\bo{B}$ has a non-trivial null space:
		%
		\begin{equation}
			\text{dim}(\text{null}(\bo{B})) = k > 0
		\end{equation}		
		%
		If the null space of $\bo{B}$ is non-trivial, then all solutions to the ID take form:
		%
		\begin{align}
			\bo{u} = \bo{B}^+ (\bo{H} \bo{a} + \bo{C}\dot{\bo{q}} + \bo{g} ) + 
			\bo{N}\bo{v}, \ \ \ \forall \bo{v} \in \R^k
			\\
			\bo{N} = \text{null}(\bo{B})
		\end{align}		
		
	\end{flushleft}
\end{frame}




\begin{frame}{Priority control input, 1}
	%\framesubtitle{How do we know the state?}
	\begin{flushleft}
		
		
		Consider a system with two inputs $\bo{u}_1$ and $\bo{u}_2$:
		%
		\begin{equation}
			\bo{H} \ddot{\bo{q}} + \bo{C}\dot{\bo{q}} + \bo{g} = \bo{B}_1\bo{u}_1 + \bo{B}_2\bo{u}_2
		\end{equation}		
		
		We would like to solve inverse dynamics without using $\bo{u}_2$ unless we have to.
		
		\textbf{Task 1}:  Solve inverse dynamics - for the current state $\bo{q}$, $\dot{\bo{q}}$ find such $\bo{u}_1$ and $\bo{u}_2$ (if it exists) that $\ddot{\bo{q}} = \bo{a}$.
		
		First we check if such pair $(\bo{u}_1, \ \bo{u}_2)$ that $\ddot{\bo{q}} = \bo{a}$ exists. We re-write the dynamics as:
		%
		\begin{equation}
			\bo{r} = 
			\begin{bmatrix}
				\bo{B}_1 & \bo{B}_2
			\end{bmatrix}
			\begin{bmatrix}
			\bo{u}_1 \\ \bo{u}_2
			\end{bmatrix}
		\end{equation}		
		
		with that, we find the criterion for the existence of the ID solution:
		%
		\begin{equation}
			\left
			(\bo{I} - \begin{bmatrix}
				\bo{B}_1 & \bo{B}_2
			\end{bmatrix}
			\begin{bmatrix}
				\bo{B}_1 & \bo{B}_2
			\end{bmatrix}^+
			\right) \bo{r} 
		= 0
		\end{equation}		
		
		
	\end{flushleft}
\end{frame}



\begin{frame}{Priority control input, 2}
	%\framesubtitle{How do we know the state?}
	\begin{flushleft}
		
		We solve the following optimization problem:
		%
		\begin{equation}
			\begin{aligned}
				& \underset{\bo{u}_1, \bo{u}_2}{\text{minimize}}
				& & \frac{1}{2} \bo{u}_2\T \bo{u}_2, \\
				& \text{subject to}
				& & \bo{r} = \bo{B}_1\bo{u}_1 + \bo{B}_2\bo{u}_2
			\end{aligned}
		\end{equation}
		
		We find Lagrangian $L = \frac{1}{2} \bo{u}_2\T \bo{u}_2 + \lambda\T (\bo{B}_1\bo{u}_1 + \bo{B}_2\bo{u}_2 - \bo{r})$.
		
		Partial derivatives of $L$ should be zero at an extremum point:
		%
		\begin{align}
			\frac{\partial L}{\partial \bo{u}_1} &=
			\lambda\T \bo{B}_1 = 0
			\\
			\frac{\partial L}{\partial \bo{u}_2} &=
			\bo{u}_2\T +
			\lambda\T \bo{B}_2 = 0
			\\
			\frac{\partial L}{\partial \lambda} &=
			\bo{B}_1\bo{u}_1 + \bo{B}_2\bo{u}_2 - \bo{r} = 0
		\end{align}
		
		
	\end{flushleft}
\end{frame}



\begin{frame}{Priority control input, 3}
	%\framesubtitle{How do we know the state?}
	\begin{flushleft}
		
		This is the same as:
		%
		\begin{equation}
			\begin{bmatrix}
				\bo{B}_1 & \bo{B}_2 & 0 \\
				0 & 0 & \bo{B}_1\T \\
				0 & \bo{I} & \bo{B}_2\T
			\end{bmatrix}
			\begin{bmatrix}
			\bo{u}_1  \\
			\bo{u}_2  \\
			\lambda 
			\end{bmatrix}
			=
			\begin{bmatrix}
			\bo{r} \\
			0  \\
			0 
			\end{bmatrix}
		\end{equation}
		
		We solve this linear system to find the desired control.
		
		
	\end{flushleft}
\end{frame}







\begin{frame}{State Space}
	%\framesubtitle{How do we know the state?}
	\begin{flushleft}
		
		Consider a state-space representation of a dynamical system:
		
		\begin{equation}
			\begin{cases}
				\dot{\bo{x}} = \bo{A}\bo{x}+\bo{B}\bo{u} 
				\\
				\bo{y} = \bo{C}\bo{x}
			\end{cases}
		\end{equation}
		
		We can introduce a change of variables $\bo{x} = \bo{T}\bo{z}$ where $\text{det}(\bo{T}) \neq 0$:
		
		\begin{equation}
			\begin{cases}
				\dot{\bo{z}} = \bo{T}^{-1}\bo{A}\bo{T} \bo{z}+\bo{T}^{-1}\bo{B}\bo{u} 
				\\
				\bo{y} = \bo{C}\bo{T}\bo{z}
			\end{cases}
		\end{equation}
		
		The two systems are equivalent, with new matrices $\bar{\bo{A}} = \bo{T}^{-1}\bo{A}\bo{T}$, $\bar{\bo{B}} = \bo{T}^{-1}\bo{B}$ and $\bar{\bo{C}} = \bo{C}\bo{T}$.
		
	\end{flushleft}
\end{frame}


\begin{frame}{State Space to ODE}
	%\framesubtitle{How do we know the state?}
	\begin{flushleft}
		
		Given a state-space representation of the form:
		%
		\begin{equation}
			\begin{cases}
				\begin{bmatrix}
					\dot x_1 \\ \dot x_2 \\  ... \\ \dot x_{n-1} \\ \dot x_n
				\end{bmatrix}
				\begin{bmatrix}
					0 & 1 & ... & 0 & 0 \\
					0 & 0& ... & 0 & 0 \\
					... & ... & ... & ... & ...  \\
					0 & 0& ... & 0 & 1 \\
					a_1 & a_2 & ... & a_{n-1} & a_n 
				\end{bmatrix}
				\begin{bmatrix}
					x_1 \\ x_2 \\  ... \\ x_{n-1} \\  x_n
				\end{bmatrix}
				+
				\begin{bmatrix}
					0 \\ 0 \\  ... \\ 0 \\  b_n
				\end{bmatrix}
				u
				\\
				y = 
				\begin{bmatrix}
					1 &0 &  ... &0 &  0
				\end{bmatrix}
				\begin{bmatrix}
				x_1 \\ x_2 \\  ... \\ x_{n-1} \\  x_n
				\end{bmatrix}
			\end{cases}
		\end{equation}
		
		We can re-write it in an ODE form:
		%
		\begin{equation}
			y^{(n)} = a_1 y + a_2 \dot y + ... + a_n y^{(n-1)} + b_n u
		\end{equation}
		
		
	\end{flushleft}
\end{frame}


\begin{frame}{State Space to ODE, 1}
	%\framesubtitle{How do we know the state?}
	\begin{flushleft}
		
		As long as state-space system has matrices:
		%
		\begin{align}
			\bar{\bo{A}} = 
			\begin{bmatrix}
				\bo{0}_{(n-1) \times 1} & \bo{I}_{(n-1) \times (n-1)} \\
				\alpha & \beta\T
			\end{bmatrix},
			\ \ \
			\bar{\bo{B}} = 
			\begin{bmatrix}
				\bo{0}_{(n-1) \times 1} \\
				1
			\end{bmatrix}
		\end{align}
		%
		we can re-write it in an ODE form. 
		
		\bigskip
		
		We can formulate a task: given matrices $\bo{A}$ and $\bo{B}$ find such transformation $\bo{T} = 
		\begin{bmatrix}
			\bo{t}_1 & ... & \bo{t}_n
		\end{bmatrix}$ 
		that $\bar{\bo{A}} = \bo{T}^{-1}\bo{A}\bo{T}$, $\bar{\bo{B}} = \bo{T}^{-1}\bo{B}$.
		
		
	\end{flushleft}
\end{frame}



\begin{frame}{State Space to ODE, 2}
	%\framesubtitle{How do we know the state?}
	\begin{flushleft}
		
		Consider the equality $\bar{\bo{B}} = \bo{T}^{-1}\bo{B}$. Since $ \bo{T}$ is full rank, we can re-write the equality:
		%
		\begin{align}
			\bo{T}\bar{\bo{B}} = \bo{B}
			\\
			\bo{t}_n = \bo{B}
		\end{align}
		
		Let us define $\bo{P} = 
		\begin{bmatrix}
			\bo{t}_1 & ... & \bo{t}_{n-1}
		\end{bmatrix}$, thus giving us $\bo{T} = 
		\begin{bmatrix}
		\bo{P}& \bo{B}
		\end{bmatrix}$.
		
		\bigskip
		
		Next, consider $\bar{\bo{A}} = \bo{T}^{-1}\bo{A}\bo{T}$:
		%
		\begin{align}
			 \bo{T}\bar{\bo{A}} = \bo{A}\bo{T}
			 \\
			 \begin{bmatrix}
				\alpha \bo{B} & (\bo{P} + \bo{B}\beta\T)
			 \end{bmatrix}
			 = 
			 \bo{A}
			 \begin{bmatrix}
			 	\bo{P}& \bo{B}
			 \end{bmatrix}
		\end{align}
		
		
		This is the solution to the posed problem. Note that we either \emph{choose a new output} $\bar{\bo{C}} = \bo{e}_1\T = [1 \ 0 \ ... \ 0]$ or original output is such that $\bo{C}\bo{T} = \bo{e}_1\T$.
		
		
	\end{flushleft}
\end{frame}





\myqrframe


\end{document}
