\documentclass{beamer}

\input{settings.tex}


\title{Fundamental subspaces}
\subtitle{\mycoursetitle, Lecture 3}
\author{by Sergei Savin}
\centering
\date{\mydate}



\begin{document}
\maketitle




\begin{frame}{Four Fundamental Subspaces}
	% \framesubtitle{Parameter estimation}
	\begin{flushleft}
		
		One of the key ideas in Linear Algebra is that every linear operator has four fundamental subspaces:
		
		\begin{itemize}
			\item Null space
			\item Row space
			\item Column space
			\item Left null space
		\end{itemize}
		
		\bigskip
		
		Our goal is to understand them. The usefulness of this concept is significant.
		
	\end{flushleft}
\end{frame}

\begin{frame}{Null space}
	\framesubtitle{Definition}
	\begin{flushleft}
		
		Consider the following task: find all solutions to the system of equations $\mathbf{A} \mathbf{x} = \mathbf{0}$.
		
		\bigskip
		
		It can be re-formulated as follows: find all elements of the \emph{null space} of $\mathbf{A}$.
		
		\begin{block}{Definition 1}
			\emph{Null space} of $\mathbf{A}$ is the set of all vectors $\mathbf{x}$ that $\mathbf{A}$ maps to $\mathbf{0}$
		\end{block}
		
		\bigskip
		
		We will denote null space as $\textcolor{mydarkgray}{\text{null}(\mathbf{A})}$. Null space of an operator is sometimes called \emph{kernel} and denoted as $\textcolor{mydarkgray}{\text{ker}(\mathbf{A})}$.
		
	\end{flushleft}
\end{frame}


\begin{frame}{Null space}
	\framesubtitle{Calculation}
	\begin{flushleft}
		
		We can find all solutions of the system of equations $\mathbf{A} \mathbf{x} = \mathbf{0}$ by using functions that generate an \emph{orthonormal basis} in the null space of $\mathbf{A}$. In MATLAB we can use the function \texttt{null}, in Python/Scipy - \texttt{null\_space}:
		
		\bigskip
		
		\begin{itemize}
			\item \texttt{N = null(A)}.
			\item \texttt{N = scipy.linalg.null\_space(A)}.
		\end{itemize}
		
		
	\end{flushleft}
\end{frame}



\begin{frame}{Null space projection}
	\framesubtitle{Local coordinates}
	\begin{flushleft}
		
		Let $\bo{N}$ be the orthonormal basis in the null space of matrix $\bo{A}$. Then, if a vector $\bo{x}$ lies in the null space of $\bo{A}$, it can be represented as:
		
		\begin{equation}
			\bo{x} = \bo{N}\bo{z}
		\end{equation}
		%
		where $\bo{z}$ are coordinates of $\bo{x}$ in the basis $\bo{N}$.
		
		\bigskip
		
		However, there are vectors which not only are not lying in the null space of $\bo{A}$,  but the closest vector to them in the null space is the zero vector.
		
	\end{flushleft}
\end{frame}


\begin{frame}{Closest element from a linear subspace}
	% \framesubtitle{Orthogonality, examples}
	\begin{flushleft}
		
		$\bo{A} = \begin{bmatrix} 0 & 1 \\ 0 & 0\end{bmatrix}$. Its null space has orthonormal basis $\bo{N} = \begin{bmatrix} 1 \\ 0\end{bmatrix}$. 
		
		\begin{itemize}
			\item $\begin{bmatrix} -2 \\ 0 \end{bmatrix} = 
			-2 \bo{N}$,
			$\begin{bmatrix} 10 \\ 0 \end{bmatrix} = 
			10 \bo{N}$, - both are in the null space.
			\item for $\bo{x} = \begin{bmatrix} 1 \\ 1 \end{bmatrix}$ the closest vector in the null space is $\begin{bmatrix} 1 \\ 0 \end{bmatrix}$.
			\item for $\bo{y} = \begin{bmatrix} 0 \\ 2 \end{bmatrix}$ the closest vector in the null space is $\begin{bmatrix} 0 \\ 0 \end{bmatrix}$.
		\end{itemize}
		
		
	\end{flushleft}
\end{frame}



\begin{frame}{Orthogonality, definition (1)}
	% \framesubtitle{Orthogonality, definition}
	\begin{flushleft}
		
		\begin{definition}
			Any two vectors, $\bo{x}$ and $\bo{y}$, whose dot product is zero are said to be \emph{orthogonal} to each other.
		\end{definition}
		
		\begin{definition}
			Vector $\bo{y}$, whose dot product with any $\bo{x} \in \mathcal{L}$ is zero is orthogonal to the subspace $\mathcal{L}$
		\end{definition}
		
		\begin{definition}[equivalent, see Appendix A]
			If for a vector $\bo{y}$, the closest vector to it from a linear subspace $\mathcal{L}$ is zero vector, $\bo{y}$ is called orthogonal to the subspace $\mathcal{L}$.
		\end{definition}
		
		
	\end{flushleft}
\end{frame}


\begin{frame}{Orthogonality, definition (2)}
	% \framesubtitle{Orthogonality, definition}
	\begin{flushleft}
		
		\begin{definition}
			The space of all vectors $\bo{y}$, orthogonal to a linear subspace $\mathcal{L}$ is called \emph{orthogonal complement} of $\mathcal{L}$ and is denoted as $\mathcal{L}^\perp$.
		\end{definition}
		
		
		\begin{definition}[equivalent]
			The space of all vectors $\bo{y}$, such that $\text{dot}(\bo{y}, \bo{x}) = 0$, $\forall \bo{x} \in \mathcal{L}$ is called \emph{orthogonal complement} of $\mathcal{L}$.
		\end{definition}
		
		Therefore $\bo{x} \in \mathcal{L}$ and $\bo{y} \in \mathcal{L}^\perp$ implies $\text{dot}(\bo{y}, \bo{x}) = 0$.
		
	\end{flushleft}
\end{frame}





\begin{frame}{Projection, 1}
	\begin{flushleft}
		
		Let $\bo{L}$ be an orthonormal basis in a linear subspace $\mathcal{L}$. Take vector $\bo{a} = \bo{x} + \bo{y}$, where $\bo{x}$ lies in the subspace $\mathcal{L}$, and $\bo{y}$ lies in the subspace $\mathcal{L}^\perp$.
		
		\bigskip
		
		\begin{definition}
			We call such vector $\bo{x}$ an \emph{orthogonal projection} of $\bo{a}$ onto subspace $\mathcal{L}$, and such vector $\bo{y}$ an orthogonal projection of $\bo{a}$ onto subspace $\mathcal{L}^\perp$
		\end{definition}
		
		\bigskip
		
		Orthogonal projection maps a vector to the element in the subspace closest to that vector. Orthogonal projection of $\bo{a}$ onto $\mathcal{L}$ can be found as: 
		
		\begin{equation}
			\bo{x} = \bo{L} \bo{L}^+ \bo{a}
		\end{equation}
		
		Since $\bo{L}$ is orthonormal, this is the same as $\bo{x} = \bo{L} \bo{L}^\top \bo{a}$
		
	\end{flushleft}
\end{frame}



\begin{frame}{Projection, 2}
	\begin{flushleft}
		
		Since $\bo{a} = \bo{x} + \bo{y}$, and $\bo{x} = \bo{L} \bo{L}^+ \bo{a}$, we can write:
		
		\begin{equation}
			\bo{a} = \bo{L} \bo{L}^+ \bo{a} + \bo{y}
		\end{equation}
		%
		from which it follows that the projection of $\bo{a}$ onto $\mathcal{L}^\perp$ can be found as: 
		
		\begin{equation}
			\bo{y} = (\bo{I} - \bo{L} \bo{L}^+) \bo{a}
		\end{equation}
		%
		where $\bo{I}$ is an identity matrix. Since $\bo{L}$ is orthonormal, this is the same as $\bo{y} = (\bo{I} - \bo{L} \bo{L}^\top) \bo{a}$
		
	\end{flushleft}
\end{frame}




\begin{frame}{Row space}
	\begin{flushleft}
		
		\begin{definition}
			Let $\mathcal{N}$ be null space of $\bo{A}$. Then orthogonal complement $\mathcal{N}^\perp$ is called \emph{row space} of $\bo{A}$.
		\end{definition}
	
		\bigskip
		
		Row space of $\bo{A}$ is the space of all smallest-norm solutions of $\bo{A}\bo{x} = \bo{y}$, for $\forall \bo{y}$. We will denote row space as $\textcolor{mydarkgray}{\text{row}(\bo{A})}$.
		
	\end{flushleft}
\end{frame}




\begin{frame}{Vectors in Null and Row spaces}
	\begin{flushleft}
		
		Given vector $\bo{x}$, matrix $\bo{A}$ and its null space basis $\bo{N}$, we check if $\bo{x}$ is in the null space of $\bo{A}$. The simplest way is to check if $\bo{A}\bo{x} = 0$. But sometimes we may want to avoid computing $\bo{A}\bo{x}$, for example if the number of elements of $\bo{A}$ is much larger than the number of elements of $\bo{N}$.
		
		\bigskip
		
		If $\bo{x}$ is in the null space of $\bo{A}$, it will have zero projection onto the row space of $\bo{A}$. This gives us the condition we can check:
		
		\begin{equation}
			(\bo{I} - \bo{N} \bo{N}^\top) \bo{x} = 0
		\end{equation}
		
		By the same logic, condition for being in the row space is as follows:
		
		\begin{equation}
			\bo{N} \bo{N}^\top \bo{x} = 0
		\end{equation}
		
		
	\end{flushleft}
\end{frame}



\begin{frame}{Column space}
	\begin{flushleft}
		
		Given a matrix $\bo{A}$ find all linear combinations of its columns: $\mathcal{C} = \{ \bo{y}: \ \bo{y} = \bo{A} \bo{x}, \ \forall \bo{x}  \}$.
		
		\bigskip
		
		It can be re-formulated as follows: find all elements of the \emph{column space} of $\bo{A}$.
		
		\begin{block}{Definition - column space}
			\emph{Column space} of $\bo{A}$ is the set of all outputs of the matrix $\bo{A}$, for all possible inputs.
		\end{block}
		
		\bigskip
		
		We will denote column space as $\textcolor{mydarkgray}{\text{col}(\bo{A})}$. It is often called an \emph{image} of $\bo{A}$.
		
		
	\end{flushleft}
\end{frame}



\begin{frame}{Column space basis}
	\begin{flushleft}
		
		The problem of finding orthonormal basis in the column space of a matrix is often called \emph{orthonormalization} of that matrix. Hence in MATLAB and Python/Scipy the function that does it is called \texttt{orth}:
		
		\bigskip
		
		\begin{itemize}
			\item \texttt{C = orth(A)}.
			\item \texttt{C = scipy.linalg.orth(A)}.
		\end{itemize}
		
		\bigskip
		
%		That is how one finds all the outputs of the matrix $\bo{A}$: as $\{ \bo{C}\bo{z}: \ \forall \bo{z} \}$. 
%		
%		Notice that $\{ \bo{A}\bo{x}: \ \forall \bo{x} \}$ might contain repeated entries if $\bo{A}$ has a non-trivial null space.
		
	\end{flushleft}
\end{frame}



\begin{frame}{Column and null spaces}
	% \framesubtitle{Local coordinates}
	\begin{flushleft}
		
		Let $\bo{A}$ be a square matrix, a map from $\mathbb{X} = \R^n$ to $\mathbb{Y} = \R^n$. 
		Notice that if it has a non-trivial null space, it follows that multiple unique inputs are being mapped by it to the same output:
		
		\begin{equation}
			\begin{aligned}
				\bo{y} = \bo{A} \bo{x}_r = \bo{A} (\bo{x}_r + \bo{x}_n), \\
				\bo{x}_r \in \text{row}(\bo{A}) \\
				\forall \bo{x}_n \in \text{null}(\bo{A}) \\
			\end{aligned}
		\end{equation}
		
		In fact, if null space of $\bo{A}$ has $k$ dimensions, it implies that an $k$-dimensional subspace of $\mathbb{X}$ is mapped to a single element of $\mathbb{Y}$. 
		
		\bigskip
		
		It follows that in this case the dimensionality of the column space could not exceed $n-k$.
		
	\end{flushleft}
\end{frame}



\begin{frame}{Projector onto column space}
	% \framesubtitle{Local coordinates}
	\begin{flushleft}
		
		Given vector $\bo{y}$ and matrix $\bo{A}$, let us find $\bo{y}_c$ - projection of $\bo{y}$ onto the column space of $\bo{A}$.
		
		\bigskip
		
		Since $\bo{y}_c \in \text{col}(\bo{A})$, we can find such $\bo{x}$ that $\bo{A}\bo{x} = \bo{y}_c$; so, the problems is to minimize the residual $e = || \bo{y}_c - \bo{y} ||$ or equivalently $e =  || \bo{A}\bo{x} - \bo{y} ||$, which is least squares problem: $\bo{x} = \bo{A}^+ \bo{y}$. So:
		
		\begin{equation}
			\bo{y}_c = \bo{A}\bo{A}^+ \bo{y} \in \text{col}(\bo{A})
		\end{equation}
		
		Remember that computing the pseudoinverse is based on SVD decomposition, same as finding a basis in the null space or the column space, so in terms of computational expense, all projections we discussed are similar.
		
	\end{flushleft}
\end{frame}



\begin{frame}{Projector onto row space}
	% \framesubtitle{Local coordinates}
	\begin{flushleft}
		
		Similarly we can define a projector onto the row space. Given vector $\bo{x}$ and matrix $\bo{A}$, let us find projector of $\bo{x}$ onto the row space of $\bo{A}$:
		
		\begin{equation}
			\bo{x}_r = \bo{A}^+\bo{A} \bo{x} \in \text{row}(\bo{A})
		\end{equation}
		
		You can think of this in the following terms: first we find output $\bo{A} \bo{x}$, then we find the smallest norm vector that produces this same output; this vector  1) has the same row space projection (because output is the same), 2) has zero null space projection. Hence it is the row space projector of $\bo{x}$.
		
		\bigskip
		
		Notice that we implicitly used the fact that columns of $\bo{A}^+$ lie in the row space of $\bo{A}$. We will prove this fact later. Additionally, we will prove that row space of $\bo{A}$ is equivalent to the column space of $\bo{A}^\top$.
		
	\end{flushleft}
\end{frame}




\begin{frame}{Left null space}
	% \framesubtitle{Local coordinates}
	\begin{flushleft}
		
		The subspace, orthogonal to the column space is called \emph{left null space}.
		
		\bigskip
		
		\begin{definition}
			Space of all vectors $\bo{y}$ orthogonal to the columns of $\bo{A}$ is called \emph{left null space}: $\bo{y}^\top\bo{A} = 0$
		\end{definition}
		
		You can think of left null space as a space of vectors that not only cannot be produced (as an output) by the operator $\bo{A}$, but the closest vector to them that can be produced is the zero vector.
		
		\bigskip
		
		Notice that $\bo{y}\T \bo{A} = 0$ implies $\bo{A}\T \bo{y} = 0$, meaning that left null space of $\bo{A}$ is equivalent to the null space of $\bo{A}\T$.
		
		
	\end{flushleft}
\end{frame}




\begin{frame}{Left null space projector}
	% \framesubtitle{Local coordinates}
	\begin{flushleft}
		
		If we want to project vector $\bo{y}$ onto the left null space of $\bo{A}$, we project it onto the column space, and subtract the result from $\bo{y}$:
		
		\begin{equation}
			\bo{y}_l = (\bo{I} - \bo{A}\bo{A}^+) \bo{y} \in \text{left null}(\bo{A})
		\end{equation}
		
		If $\bo{C}$ is an orthonormal basis in the column space of $\bo{A}$, the projection can be found the following way:
		
		\begin{equation}
			\bo{y}_l = (\bo{I} - \bo{C}\bo{C}^\top) \bo{y} \in \text{left null}(\bo{A})
		\end{equation}
		
		
	\end{flushleft}
\end{frame}





\begin{frame}{Further reading}
	% \framesubtitle{Local coordinates}
	\begin{flushleft}
		
		\begin{itemize}
			%\item \bref{https://faculty.math.illinois.edu/~mlavrov/docs/484-spring-2019/ch4lec4.pdf}{Minimum Norm Solutions, Math 484: Nonlinear Programming, Mikhail Lavrov}
			%https://faculty.math.illinois.edu/~mlavrov/courses/484-spring-2019.html
			
			\item \bref{https://faculty.math.illinois.edu/~mlavrov/docs/484-spring-2019/ch4lec3.pdf}{Orthogonality, Math 484: Nonlinear Programming, Mikhail Lavrov}
			
			\item \bref{http://databookuw.com/databook.pdf}{Data Driven Science \& Engineering. Machine Learning, Dynamical Systems, and Control, Steven L. Brunton, J. Nathan Kutz}, chapter Singular Value Decomposition (SVD)
			
		\end{itemize}
		
		
	\end{flushleft}
\end{frame}




\begin{frame}{Exercise}
	% \framesubtitle{Local coordinates}
	\begin{flushleft}
		
		\begin{itemize}
			\item Matrix $\bo{M}$ is orthonormal and square, prove that $\bo{M}\T = \bo{M}^{-1}$.
			
			\item Find minimum of $||\bo{A}\bo{x} - \bo{y}||_2$ when columns of $\bo{A}$ are not linearly independent.
			
			\item Given an equation $\bo{A}\bo{x} = \bo{y}$ with a square matrix $\bo{A}$, prove that: either that equation has an exact solution for any $\bo{y}$ or a related homogeneous equation $\bo{A}\bo{x} = 0$ has a non-trivial solution.
		\end{itemize}
		
		
		
	\end{flushleft}
\end{frame}




\myqrframe



\begin{frame}{Appendix A}
	% \framesubtitle{Local coordinates}
	\begin{flushleft}
		
		We have two definitions of orthogonality of a vector and a subspace:
		
		\begin{enumerate}
			\item Vector $\bo{y}$, whose dot product with any $\bo{x} \in \mathcal{L}$ is 0 is orthogonal to the subspace $\mathcal{L}$
			
			\item If for a vector $\bo{y}$, the closest vector to it from a linear subspace $\mathcal{L}$ is zero vector, $\bo{y}$ is called orthogonal to the subspace $\mathcal{L}$.
		\end{enumerate}
		
	Let us prove their equivalence. First we show that 1) implies 2). Let $\bo{L}$ be orthonormal basis in $\mathcal{L}$. To find the closest element $\bo{y}^*$ of $\mathcal{L}$ to $\bo{y}$, we need to solve the least squares problem $\bo{L}\bo{z} = \bo{y}$, and multiply the solution by $\bo{L}$:
	
	\begin{align}
		\bo{z}_{LS} &= \bo{L}\T \bo{y} = \bo{0} \\
    	\bo{y}^* = \bo{L}\bo{z}_{LS} &= \bo{L}\bo{L}\T \bo{y} = \bo{0}
	\end{align}		
		
	\end{flushleft}
\end{frame}


\begin{frame}{Appendix A}
	% \framesubtitle{Local coordinates}
	\begin{flushleft}
		
		
		Second, let us prove that 2) implies 1). Given that $\bo{y}^* = \bo{L}\bo{z}_{LS} = \bo{L}\bo{L}\T \bo{y} = \bo{0}$ we need to prove that $\bo{L}\T \bo{y} = \bo{0}$. We start by multiplying $\bo{L}\bo{L}\T \bo{y} = \bo{0}$ by $\bo{L}\T$:
		
		\begin{align}
			& \bo{L}\T \bo{L}\bo{L}\T \bo{y} = \bo{0} \\
			& \bo{L}\T \bo{y} = \bo{0}  \ \ \ \ \ \ \textcolor{mygray}{\text{since }  \bo{L}\T \bo{L} = \bo{I}}. \ \qed
		\end{align}		
		
	\end{flushleft}
\end{frame}





\end{document}
