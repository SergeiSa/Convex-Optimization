\documentclass{beamer}

\input{settings.tex}
\usepackage{multicol}


\title{Linear inequalities and polytopes}
\subtitle{\mycoursetitle, Lecture 6}
\author{by Sergei Savin}
\centering
\date{\mydate}

\begin{document}
\maketitle


\begin{frame}{Content}

\begin{itemize}
\item Convex polytopes
\item Half-spaces
\item H-representation
\item V-representation
\item G-representation (Zonotopes)
\item Linear approximation of convex regions
\end{itemize}

\end{frame}



\begin{frame}{Convex polytopes}
% \framesubtitle{Parameter estimation}
\begin{flushleft}

Before defining what a convex polytope is, let us look at examples:

\include{fig1}
 
\end{flushleft}
\end{frame}


\begin{frame}{Convex polytopes}
% \framesubtitle{Parameter estimation}
\begin{flushleft}

You can think of polytopes as geometric figures (or continuous sets of points) with linear edges, faces and higher-dimensional analogues.

\bigskip

\begin{definition}
 Convex polytopes are polytopes whose every two points can be connected with a line that would lie in the polytope. They can be bounded or unbounded.
\end{definition}
 
\end{flushleft}
\end{frame}


\begin{frame}{Half-spaces}
\framesubtitle{Definition}
\begin{flushleft}

We can define half-space as a set of all points $\mathbf{x}$, such that $\mathbf{a}^\top \mathbf{x} \leq b$. It has a very clear geometric interpretation. In the following image, the filled space is \textbf{not} in the half space.

\include{fig2}
 
\end{flushleft}
\end{frame}



\begin{frame}{Half-spaces}
\framesubtitle{Construction. Simple case}
\begin{flushleft}

Consider a half-space that passes through the origin, and defined by its normal vector $\mathbf{n}$:

\include{fig3}

This half-space can be defined as "all vectors $\mathbf{x}$, such that $\mathbf{n} \cdot \mathbf{x} \leq 0$", which is the same as using $\mathbf{n}$ instead of $\mathbf{a}$ in our original definition, setting $b = 0$.
 
\end{flushleft}
\end{frame}




\begin{frame}{Half-spaces}
\framesubtitle{Construction. General case}
\begin{flushleft}

In the general case there is some distance between the boundary of the half-space and the origin, let's say $d$.

\include{fig4}
%
Here the half space can be defined as "all vectors $\mathbf{x}$, such that $\mathbf{x}^\top \frac{\mathbf{n}}{|| \mathbf{n} ||}  \leq d$". This is the same as making $\mathbf{a} = \mathbf{n}$ and $b = d ||\mathbf{n}||$.
 
\end{flushleft}
\end{frame}



\begin{frame}{Half-spaces}
\framesubtitle{Combination}
\begin{flushleft}

We can define a region of space as an \emph{intersection} of half-spaces $\mathbf{a}_i^\top \mathbf{x} \leq b_i$:

\include{fig5}

Resulting region will be easily described as $\begin{bmatrix} \mathbf{a}_1^\top \\ ... \\ \mathbf{a}_k^\top \end{bmatrix} \mathbf{x} \leq \begin{bmatrix} b_1 \\ ... \\ b_k \end{bmatrix}$

 
\end{flushleft}
\end{frame}


\begin{frame}{H-representation of a polytope}
%\framesubtitle{}
\begin{flushleft}

The last result allows us to write any convex polytope as a matrix inequality:

\begin{equation}
\label{eq:ineq} 
    \mathbf{A} \mathbf{x} \leq  \mathbf{b} 
\end{equation}

And conversely, any matrix inequality \eqref{eq:ineq} represents either an empty set or a convex polytope.

\bigskip

\begin{definition}
 $\mathbf{A} \mathbf{x} \leq  \mathbf{b}$ is called \emph{H-representation} (half-space representation) of a polytope.
\end{definition}
 
\end{flushleft}
\end{frame}



\begin{frame}{H-representation in COP}
	%\framesubtitle{}
	\begin{flushleft}
		
		We can use containment in an H-polytope as a part of convex optimation problem. For example, the following QP includes such constraint:
		
		\begin{equation}
			\begin{aligned}
				& \underset{\mathbf{x}}{\text{minimize}}
				& & \mathbf{x}^\top \mathbf{H} \mathbf{x} + \mathbf{f}^\top\mathbf{x}, \\
				& \text{subject to}
				& & \mathbf{A}\mathbf{x} \leq \mathbf{b}.
			\end{aligned}
		\end{equation}
		
	\end{flushleft}
\end{frame}



\begin{frame}{Example}
	%\framesubtitle{}
	\begin{flushleft}
		
		Find robot's position $\bo{r} = [x, \ y]\T$ in a room of length $l$ and width $w$, closest to the desired point $\bo{r}_d= [x_d, \ y_d]\T$; the coordinate system is aligned with axes of the room, the center of the room is the point (0, 0).
		
		\bigskip
		
		\hspace{-0.2in}
		\textbf{Solution}. The geometry of the room imposes 4 constraints on $\bo{r}$:
		
		\begin{multicols}{2}
		\begin{enumerate}
			\item $\begin{bmatrix} 1 & 0 \end{bmatrix} \bo{r} \leq 0.5 l$
			\item $\begin{bmatrix} -1 & 0 \end{bmatrix}  \bo{r} \leq 0.5 l$
			\item $\begin{bmatrix} 0 & 1 \end{bmatrix}  \bo{r} \leq 0.5 w$
			\item $\begin{bmatrix} 0 & -1 \end{bmatrix}  \bo{r} \leq 0.5 w$
		\end{enumerate}
	\end{multicols}
		
		The solution takes the form of a QP:
		
		\begin{equation}
			\begin{aligned}
				& \underset{x, y}{\text{minimize}}
				& & (x-x_d)^2 + (y-y_d)^2, \\
				& \text{subject to}
				& & \begin{bmatrix} 1 & 0 \\ -1 & 0 \\ 0 & 1 \\ 0 & -1 \end{bmatrix}
				\begin{bmatrix} x \\ y \end{bmatrix}
				\leq
				\begin{bmatrix} 0.5 l \\ 0.5 l \\ 0.5 w \\ 0.5 w \end{bmatrix}
			\end{aligned}
		\end{equation}
		
	\end{flushleft}
\end{frame}





\begin{frame}{V-representation}
% \framesubtitle{}
\begin{flushleft}

Convex polytopes have alternative representations, such as \emph{V-representation}. It amounts to representing polytope as a set of its vertices.

\begin{example}
$V = \begin{bmatrix} -1 & -1 & 1 & 1 \\ -1 & 1 & 1 & -1 \end{bmatrix}$ is a V-representation of a square.
\end{example}

\begin{example}
$\begin{bmatrix} 1 & 0 \\ 0 & 1 \\ -1 & 0 \\ 0 & -1 \end{bmatrix}
\begin{bmatrix} x_1 \\ x_2 \end{bmatrix} \leq
\begin{bmatrix} 1 \\ 1 \\ 1 \\ 1 \end{bmatrix}$
is an H-representation of the same square.
\end{example}
 
\end{flushleft}
\end{frame}


\begin{frame}{Convex hull}
	% \framesubtitle{Parameter estimation}
	\begin{flushleft}
		
		Given points $\bo{x}_1$, $\bo{x}_2$, ..., $\bo{x}_N$ their convex hull is represented as:
		
		\begin{equation}
			\mathcal{P} = \left \{  \bo{x} = \sum_{i=1}^{N} \alpha_i\bo{x}_i: \   \sum_{i=1}^{N} \alpha_i = 1, \  \alpha_i  \in [0 \ 1] \right \}
		\end{equation}
		
		See Appendix for an illustration of this formula.
		
	\end{flushleft}
\end{frame}



\begin{frame}{V-representation in COP}
	%\framesubtitle{}
	\begin{flushleft}
		
		We can use containment in an V-polytope as a part of convex optimation problem. For example, the following QP includes such constraint:
		
		\begin{equation}
			\begin{aligned}
				& \underset{\mathbf{x}}{\text{minimize}}
				& & \mathbf{x}^\top \mathbf{H} \mathbf{x} + \mathbf{f}^\top\mathbf{x}, \\
				& \text{subject to}
				& & \begin{cases}
					\mathbf{x} = \sum\limits_{i=1}^{n} \alpha_i  \mathbf{v}_i, \\
					\sum\limits_{i=1}^{n} \alpha_i = 1, \\
					\alpha_i \geq 0.
				\end{cases}
			\end{aligned}
		\end{equation}
		
		Notice that the constraint amounts to equating $\mathbf{x}$ to a convex combination of the vertices of the V-polytope.
		
	\end{flushleft}
\end{frame}





\begin{frame}{Example}
	%\framesubtitle{}
	\begin{flushleft}
		
		Find robot's position $\bo{r} = [x, \ y]\T$  closest to the desired point $\bo{r}_d= [x_d, \ y_d]\T$ in a fenced-off area. The fence connects point $(-10, 0)$ to $(0, 10)$ to $(7, 7)$ to $(0, -10)$.
		
		\bigskip
		
		\textbf{Solution}:
		
		\begin{equation}
			\begin{aligned}
				& \underset{x, y}{\text{minimize}}
				& & (x-x_d)^2 + (y-y_d)^2, \\
				& \text{subject to}
				& & 
				\begin{cases}
					\begin{bmatrix} x \\y \end{bmatrix}
					=
					\alpha_1
					\begin{bmatrix} -10 \\0 \end{bmatrix}
					+
					\alpha_2
					\begin{bmatrix} 0 \\10 \end{bmatrix}
					+
					\alpha_3
					\begin{bmatrix} 7 \\7 \end{bmatrix}
					+
					\alpha_4
					\begin{bmatrix} 0 \\ -10 \end{bmatrix}
					\\
					\alpha_1 + \alpha_2 + \alpha_3 + \alpha_4 = 1
					\\
					\alpha_1 \geq 0 \\  \alpha_2 \geq 0 \\  \alpha_3 \geq 0 \\  \alpha_4 \geq 0
				\end{cases}
			\end{aligned}
		\end{equation}
		
	\end{flushleft}
\end{frame}




\begin{frame}{H and V-representations}
% \framesubtitle{}
\begin{flushleft}

To transfer from H-representation to V-representation, you need to solve \emph{vertex enumeration} problem, which is computationally expensive. 

\bigskip

It is also possible to recover H-representation from V-representation. 
 
\end{flushleft}
\end{frame}




\begin{frame}{Zonotopes: G-representation}
	% \framesubtitle{}
	\begin{flushleft}
		
		A zonotope $\mathcal{Z}$ is a symmetric polytope defined by its \emph{center} $\bo{c}$ and \emph{generator} $\bo{G}$:
		
		\begin{equation}
			\mathcal{Z} = \{ \bo{x}: \ \bo{x}=\bo{G}\beta+\bo{c}, \ ||\beta||_\infty \leq 1  \}
		\end{equation}
	
		The set $\{ \beta: \ ||\beta||_\infty \leq 1  \}$ is a hypercube and zonotope $\mathcal{Z}$ is a projection (shadow) of this hypercube onto a lower-dimensional space; the projection is defined by the matrix $\bo{G}$.
		
		% TODO: \usepackage{graphicx} required
		\begin{figure}
			\centering
			\includegraphics[width=0.4\linewidth]{zonotope_example}
			\caption{Zonotope (\bref{https://www.researchgate.net/publication/322671928_Methods_for_order_reduction_of_zonotopes}{Source}) }
			\label{fig:zonotopeexample}
		\end{figure}
		%https://www.researchgate.net/publication/322671928_Methods_for_order_reduction_of_zonotopes
		
		
	\end{flushleft}
\end{frame}




\begin{frame}{G-representation in COP}
	%\framesubtitle{}
	\begin{flushleft}
		
		We can use containment in an G-polytope as a part of convex optimation problem. For example, the following QP includes such constraint:
		
		\begin{equation}
			\begin{aligned}
				& \underset{\mathbf{x}}{\text{minimize}}
				& & \mathbf{x}^\top \mathbf{H} \mathbf{x} + \mathbf{f}^\top\mathbf{x}, \\
				& \text{subject to}
				& & \begin{cases}
					\mathbf{x} = \bo{G}\beta+\bo{c}, \\
					-1 \leq \beta_i \leq 1.
				\end{cases}
			\end{aligned}
		\end{equation}
		
	\end{flushleft}
\end{frame}




\begin{frame}{Linear approximation of convex regions}
% \framesubtitle{Parameter estimation}
\begin{flushleft}
Some convex regions can be easily approximated using polytopes.

\include{fig6}
%
This allows us to represent constraints on $\mathbf{x}$ to belong in such a region as a matrix inequality
 
\end{flushleft}
\end{frame}



\begin{frame}{Exercise}
% \framesubtitle{Parameter estimation}
\begin{flushleft}

Write H-representation of the following polytopes:

\begin{itemize}
    \item Equilateral triangle
    \item Square
    \item Parallelepiped
    \item Trapezoid
\end{itemize}

\end{flushleft}
\end{frame}


% \begin{frame}{Self-study}
% % \framesubtitle{Part 3}
% \begin{flushleft}

% \begin{itemize}
%     \item \href{https://www.youtube.com/watch?v=kcOodzDGV4c}{Convex Optimization, lecture 3, S. Boyd. Stanford. Convex functions}.
% \end{itemize}

% \end{flushleft}
% \end{frame}

\myqrframe





\begin{frame}{Appendix A - convex hull, 1}
	% \framesubtitle{Parameter estimation}
	\begin{flushleft}
		
		
		\begin{figure}[.45\textwidth]
			\centering
			\resizebox{.65\textwidth}{!}
			{\input{figure_convhull_0}}
		\end{figure}
		
		Let us illustrate the convex combination formula. 
		Let $\mathcal{P}$ be convex hull of points $\bo{x}_1$, $\bo{x}_2$ and $\bo{x}_3$:
		
		
		\begin{equation}
			\mathcal{P} = \left \{  \bo{x} = \sum_{i=1}^{3} \alpha_i\bo{x}_i: \   \sum_{i=1}^{3} \alpha_i = 1, \  \alpha_i  \in [0 \ 1] \right \}
		\end{equation}
		
		Let $\bo{z}$ be a convex combination of $\bo{x}_1$ and $\bo{x}_3$: $\bo{z} = \beta_1 \bo{x}_1 + \beta_3 \bo{x}_3 $. Any point $\bo{p} \in \mathcal{P}$ can be defined as a convex combination of correctly chosen $\bo{z}$ and $\bo{x}_2$: $\bo{p} = \gamma_1 \bo{z} + \gamma_2 \bo{x}_2$.
		
	\end{flushleft}
\end{frame}



\begin{frame}{Appendix A - convex hull, 2}
	% \framesubtitle{Parameter estimation}
	\begin{flushleft}
		
		
		
		\begin{figure}[.45\textwidth]
			\centering
			\resizebox{.45\textwidth}{!}
			{\input{figure_convhull_0}}
		\end{figure}
		
		We can express $\bo{p}$ as:
		%
		\begin{align}
			\bo{p} =  \gamma_1 \bo{z} + \gamma_2 \bo{x}_2 
			           = \gamma_1 (\beta_1 \bo{x}_1 + \beta_3 \bo{x}_3 )+ \gamma_2 \bo{x}_2
		\end{align}
		%
		We can define $\alpha_1 = \gamma_1 \beta_1$, $\alpha_2= \gamma_2 \bo{x}_2$ and $\alpha_3 = \gamma_1 \beta_3$. Since $\gamma_i \geq 0$ and $\beta_i \geq 0$, we conclude that $\alpha_i \geq 0$.
		
		\bigskip
		
		We can show that $e = \alpha_1 + \alpha_2+\alpha_3 = 1$:
		%
		\begin{align}
			e =  \gamma_1(\beta_1 +\beta_3) + \gamma_2 = \gamma_1 + \gamma_2 = 1
		\end{align}
		
		
	\end{flushleft}
\end{frame}




\begin{frame}{Appendix A - convex hull, 3}
	% \framesubtitle{Parameter estimation}
	\begin{flushleft}
		
		
					\begin{figure}
							\begin{subfigure}{.5\textwidth}
									\centering
									\resizebox{.95\textwidth}{!}
									{\input{figure_convhull_1}}
%									\label{fig:sfig1}
								\end{subfigure}%
							\begin{subfigure}{.5\textwidth}
									\centering
									\resizebox{.95\textwidth}{!}
									{\input{figure_convhull_2}}
%									\label{fig:sfig2}
								\end{subfigure}
%							\caption{plots of....}
%							\label{fig:fig}
						\end{figure}
		
		Previously we illustrated sufficiency of the formula's constraints. Now let us illustrate their necessity.
		
		\bigskip
		
		Dropping requirement $\alpha_i \leq 1$, and/or $\alpha_i \geq 0$ and/or $\sum\limits_{i=1}^{3} \alpha_i = 1,$ leads to inclusion of points out the convex hull, as illustrated on the figures.
		
		
	\end{flushleft}
\end{frame}




\end{document}
